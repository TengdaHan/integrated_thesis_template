%%% BIBLIOGRAPHY SETUP
%%% Originally from https://github.com/twgr/thesis,
%%% heavily tweaked by Shangzhe Wu (szwu@robots.ox.ac.uk),
%%% slightly modified by Tengda Han (htd@robots.ox.ac.uk).

% Note that your bibliography will require some tweaking depending on your department, preferred format, etc.
% The options included below are just very basic "sciencey" and "humanitiesey" options to get started.
% If you've not used LaTeX before, I recommend reading a little about biblatex/biber and getting started with it.
% If you're already a LaTeX pro and are used to natbib or something, modify as necessary.
% Either way, you'll have to choose and configure an appropriate bibliography format...

% The science-type option: numerical in-text citation with references in order of appearance.
% \usepackage[style=numeric-comp, sorting=none, backend=biber, doi=false, isbn=false]{biblatex}
% \newcommand*{\bibtitle}{References}

% The humanities-type option: author-year in-text citation with an alphabetical works cited.
%\usepackage[style=authoryear, sorting=nyt, backend=biber, maxcitenames=2, useprefix, doi=false, isbn=false]{biblatex}
%\newcommand*{\bibtitle}{Works Cited}

\usepackage[style=authoryear, sorting=nyt, backend=biber, maxcitenames=2, maxbibnames=99, useprefix, firstinits=false, dashed=false, doi=false, isbn=false, uniquelist=false]{biblatex}
\DeclareNameAlias{sortname}{first-last}
% \usepackage[style=ieee, sorting=nyt, backend=biber, firstinits=false, useprefix, doi=false, isbn=false]{biblatex}
\newcommand*{\bibtitle}{References}

% change parentheses in citations into square brackets, use \parencite{} (https://tex.stackexchange.com/questions/16765/biblatex-author-year-square-brackets)
\makeatletter
\newrobustcmd*{\parentexttrack}[1]{%
  \begingroup
  \blx@blxinit
  \blx@setsfcodes
  \blx@bibopenparen#1\blx@bibcloseparen
  \endgroup}

\AtEveryCite{%
  \let\parentext=\parentexttrack%
  \let\bibopenparen=\bibopenbracket%
  \let\bibcloseparen=\bibclosebracket}
\makeatother

% hyperref both name and year in citations (https://tex.stackexchange.com/questions/15951/hyperlink-name-with-biblatex-authoryear-biblatex-1-4b)
\DeclareFieldFormat{citehyperref}{%
  \DeclareFieldAlias{bibhyperref}{noformat}% Avoid nested links
  \bibhyperref{#1}}

\DeclareFieldFormat{textcitehyperref}{%
  \DeclareFieldAlias{bibhyperref}{noformat}% Avoid nested links
  \bibhyperref{%
    #1%
    \ifbool{cbx:parens}
      {\bibcloseparen\global\boolfalse{cbx:parens}}
      {}}}

\savebibmacro{cite}
\savebibmacro{textcite}

\renewbibmacro*{cite}{%
  \printtext[citehyperref]{%
    \restorebibmacro{cite}%
    \usebibmacro{cite}}}

\renewbibmacro*{textcite}{%
  \ifboolexpr{
    ( not test {\iffieldundef{prenote}} and
      test {\ifnumequal{\value{citecount}}{1}} )
    or
    ( not test {\iffieldundef{postnote}} and
      test {\ifnumequal{\value{citecount}}{\value{citetotal}}} )
  }
    {\DeclareFieldAlias{textcitehyperref}{noformat}}
    {}%
  \printtext[textcitehyperref]{%
    \restorebibmacro{textcite}%
    \usebibmacro{textcite}}}

\renewcommand*{\cite}[1]{\parencite{#1}}
% \renewbibmacro*{cite}{
% 	\usebibmacro{parencite}
% }

% This makes the bibliography left-aligned (not 'justified') and slightly smaller font.
\renewcommand*{\bibfont}{\raggedright\small}

\AtEveryBibitem{\clearfield{volume}\clearfield{number}\clearfield{page}\clearfield{pages}\clearfield{organization}\clearfield{organisation}}

% add bib source here
\addbibresource{vgg_bibtex/shortstrings.bib}
\addbibresource{chapter/paper1/egbib.bib}
\addbibresource{vgg_bibtex/vgg_other.bib}
\addbibresource{vgg_bibtex/vgg_local.bib}

%%% BIBLIOGRAPHY SETUP finishes here %%%



%%% GENERAL
\newcommand{\centre}[1]{\begin{center} #1 \end{center}}
\newcommand{\HRule}{\rule{\linewidth}{0.5mm}}
\newcommand{\bs}{\symbol{34}}

\DeclareUnicodeCharacter{00A0}{ }

\let\pagebreakORIG\pagebreak
\let\clearpageORIG\clearpage
\let\cleardoublepageORIG\cleardoublepage

\ifx \removepagebreak \undefined
\newcommand{\removepagebreak}{\renewcommand{\pagebreak}{}\renewcommand{\clearpage}{}\renewcommand{\cleardoublepage}{}}
\fi

\ifx \restorepagebreak \undefined
\newcommand{\restorepagebreak}{\renewcommand{\pagebreak}{\pagebreakORIG}\renewcommand{\clearpage}{\clearpageORIG}\renewcommand{\cleardoublepage}{\cleardoublepageORIG}}
\fi
% right aside information
%\usepackage{background}
%\newcommand\Text{\tiny Dassault Systèmes | Confidential Information}
%\SetBgColor{black}
%\SetBgOpacity{1}
%\SetBgAngle{90}
%\SetBgPosition{current page.center}
%\SetBgVshift{-0.60\textwidth}
%\SetBgScale{1.0}
%\SetBgContents{\sffamily\Text}
\renewcommand{\headrulewidth}{0pt}
\fancyhfoffset[R]{50pt}

\makeatletter
\DeclareRobustCommand\onedot{\futurelet\@let@token\@onedot}
\def\@onedot{\ifx\@let@token.\else.\null\fi\xspace}
\def\cf{\emph{c.f}\onedot} \def\Cf{\emph{C.f}\onedot}
\def\cf{\emph{c.f}\onedot} \def\Cf{\emph{C.f}\onedot}
\def\etc{\emph{etc}\onedot} \def\vs{\emph{vs}\onedot}
\def\wrt{w.r.t\onedot} \def\dof{d.o.f\onedot}
\makeatother

\makeatletter
\let\ps@plain=\ps@fancy
\makeatother

\providecommand{\eg}[0]{e.g\xperiod}
\providecommand{\ie}[1]{i.e\xperiod}
\newcommand{\etal}{\textit{et al}.}
%\providecommand{\etal}[2]{et al.\xperiod}



%%% Background Picture
\usepackage{eso-pic}
\newcommand\BackgroundPic{%
	\put(320,5){%
		\parbox[b][1cm]{\paperwidth}{%
			\vfill
			\centering
			%\includegraphics[width=\paperwidth,height=1cm,%
			%keepaspectratio]{footer_ox.png}%
}}}


%%% Algorithm
\algdef{SE}[SUBALG]{Indent}{EndIndent}{}{\algorithmicend\ }%
\algtext*{Indent}
\algtext*{EndIndent}

\renewcommand{\baselinestretch}{1.5}
\lstset{
	language=c,
	numbers=left,
	stepnumber=1,
	numbersep=10pt,
	frame=leftline,
	title=\lstname,                   % show the filename of files included with \lstinputlisting;
	% also try caption instead of title
	keywordstyle=\color{blue},          % keyword style
	commentstyle=\color{dkgreen},       % comment style
	stringstyle=\color{mauve},         % string literal style
	escapeinside={\%*}{*)},            % if you want to add LaTeX within your code
	morekeywords={*,...},              % if you want to add more keywords to the set
	deletekeywords={...}             % if you want to delete keywords from the given language
}

\lstset{ literate={á}{{\'a}}1 {é}{{\'e}}1 {í}{{{\'\i}}}1 {ó}{{\'o}}1 {ö}{{\"o}}1 {ő}{{\H o}}1 {ú}{{\'u}}1 {Ú}{{\'U}}1 {ü}{{\"u}}1 {ű}{{\H u}}1 {Ü}{{\"U}}1 }

\lstset{language=C}
\lstdefinestyle{customc}{
	belowcaptionskip=1\baselineskip,
	breaklines=true,
	frame=L,
	xleftmargin=\parindent,
	language=C,
	showstringspaces=false,
	basicstyle=\footnotesize\ttfamily,
	identifierstyle=\color{black},
	stringstyle=\color{orange},
}
\lstset{escapechar=@,style=customc}



%%% Custom colors
\definecolor{dkgreen}{rgb}{0,0.6,0}
\definecolor{gray}{rgb}{0.5,0.5,0.5}
\definecolor{mauve}{rgb}{0.58,0,0.82}
\definecolor{deepblue}{rgb}{0,0,0.5}
\definecolor{deepred}{rgb}{0.6,0,0}
\definecolor{deepgreen}{rgb}{0,0.5,0}
\definecolor{codegreen}{rgb}{0,0.6,0}
\definecolor{codegray}{rgb}{0.5,0.5,0.5}
\definecolor{codepurple}{rgb}{0.58,0,0.82}
\definecolor{backcolour}{rgb}{0.95,0.95,0.92}
\definecolor{codeblue}{rgb}{0.25,0.5,0.5}

\definecolor{mygreen}{RGB}{34,139,34}
\definecolor{myblue}{rgb}{0.19, 0.55, 0.91}
\definecolor{googleblue}{RGB}{66,133,244}
\definecolor{googleorange}{RGB}{255,171,64}
\definecolor{oxblue}{RGB}{0,33,71}
\definecolor{tcolor}{RGB}{0,71,153}
% \definecolor{lcolor}{RGB}{76,126,183}
\definecolor{lcolor}{RGB}{102,144,193}
\definecolor{teal}{RGB}{0,128,128}

\lstset{
	backgroundcolor=\color{white},
	basicstyle=\fontsize{8.2pt}{8.2pt}\ttfamily\selectfont,
	columns=fullflexible,
	breaklines=true,
	captionpos=b,
	commentstyle=\color{codeblue},
	keywordstyle=\color{codegreen},
	stringstyle=\color{codepurple}
	%  frame=tb,
}
% \usepackage[dvipsnames]{xcolor}
\definecolor{mygray}{gray}{0.4}
\newcommand{\app}[0]{{Appendix}}



% MISC from papers
\usepackage{dsfont}
\usepackage{bm}
\usepackage{parskip}
\usepackage{cleveref}
\usepackage{textcomp}
\usepackage{dblfloatfix}    
\usepackage{wrapfig,booktabs}
\usepackage{booktabs}
\usepackage{tabularx}
\usepackage{multicol}
\usepackage{enumitem}
\usepackage{calc}
\usepackage{dblfloatfix}    
\def\viz{\emph{viz}\onedot}
% \setlength{\tabcolsep}{4.5pt}

\newcommand{\textlongs}{{\fontencoding{TS1}\fontfamily{lmr}\selectfont\char115}}
\def\viz{\emph{viz}\onedot}
\newcommand{\bx}{\bm{x}}
\newcommand{\bp}{\bm{p}}
\newcommand{\bg}{\bm{g}}
\newcommand{\bw}{\bm{w}}
\newcommand{\by}{\bm{y}}
\newcommand{\Ex}{\mathop{\mathbb{E}}}

\newcommand{\attn}[1]{\textbf{\color{orange}attn\@:~#1}}
\newcommand{\TODO}[1]{\textcolor{red}{#1}}
\newcommand{\floor}[1]{\left\lfloor #1 \right\rfloor}
\newcommand{\norm}[1]{\left\lVert #1 \right\rVert}
\newcommand{\matr}[1]{\mathbf{#1}}

\newcommand{\reals}[1]{\mathbb{R}^{#1}}
\newcommand{\set}[1]{\left\{{#1}\right\}}
\newcommand{\transpose}[1]{#1^\top}
\newcommand*\rot{\rotatebox{90}}

\newcommand{\seqset}{\mathcal{S}}
\newcommand{\enorm}[1]{\left\|{#1}\right\|_2}
\DeclareMathOperator{\loss}{loss}
\newcommand{\binary}[1]{\mathbb{B}^{#1}}
\newcommand{\one}{\mathbf{1}}

% \DeclareMathOperator*{\argmax}{argmax}   % argmax
% \DeclareMathOperator*{\argmin}{argmin}   % argmin
\DeclareMathAlphabet{\mathpzc}{OT1}{pzc}{m}{it}
\providecommand{\etc}[0]{\emph{etc.}}
\providecommand{\ie}[0]{\emph{i.e\xperiod}}
\providecommand{\eg}[0]{\emph{e.g\xperiod}}
\providecommand{\etal}[0]{\emph{et al\xperiod}}
\DeclareFixedFont{\ttb}{T1}{txtt}{bx}{n}{6} % for bold
\DeclareFixedFont{\ttm}{T1}{txtt}{m}{n}{6}  % for normal

\hypersetup{
	colorlinks=true,
	pdfborder={0 0 0},
	citecolor=teal,
}



%%% APPENDIX
% JEM: combine all of the commands you run before appendices start
\newcommand{\startappendices}
  {\appendix\adjustmtc\noappendicestocpagenum\addappheadtotoc\appendixpage}



%%% EASY REFERENCE
%%% HTD: To add paper from original directory, define the scope for \ref and \label command
\def\currentprefix{main:}
\def\globalprefix{main:}
\let\fixedref\ref
\let\fixedlabel\label
\let\fixedcref\cref
\newcommand{\globalref}[1]{\fixedref{\globalprefix:#1}}

% macro to define a local label
\renewcommand*{\label}[1]{\fixedlabel{\currentprefix:#1}}
% macro to use a local reference
\renewcommand*{\ref}[1]{\fixedref{\currentprefix:#1}}
\renewcommand*{\cref}[1]{\fixedcref{\currentprefix:#1}}

% relative \input command
\let\fixedinput\input 
\let\fixedincludegraphics\includegraphics 
